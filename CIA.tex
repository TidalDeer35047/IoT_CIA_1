\documentclass[12pt, letterpaper]{article}
\title{Paper review on 'A Review of IoT Sensing Applications and Challenges
Using RFID and Wireless Sensor Networks'}
\author{Akshat K - 21011101015}
\begin{document}
\maketitle
\textbf{Summary}
\\ \ \\
The paper focuses mainly on two technologies in wide use in the present age, namely, RFID (Radio Frequency Technology)
and WSN(Wireless Sensor Networks). These two technologies together are the 'two fundamental pillars' of modern IoT applications 
in daily life.They are simple but smart objects that are utilized for their wireless communication and sensing capabilities, both 
with their own use-cases.
\\ \ \\
\textbf{RFID}
\\ \ \\
RFID is an auto identification technology that uses two main types of devices: a reader, which is
thepart that facilitates communication, and the tags, which have an associated electronic code they use 
to  have a unique identity to be identified. The reader communicates with these tags using radio frequency (RF) signals,
and the tags respond with their identification code (ID). Tags may also incorporate a sensor, in which
case they will scatter the data from their sensor. Tags can be active (powered by a battery) or
passive (reaping the energy from the reader's RF signal). RFID is a consolidated technology for the
identification of assets, security, and track-and-trace applications with a high and increasing density of
tags available for its own specialized utility.
\\ \ \\
A rapid growth of RFID technology for identification and tracking
applications has been witnessed for the past couple of decades. The ability of RFID to 
identify, trace, and track information using easily deployable tags is
now enabling applications have abilities beyond their traditional equivalents, it is now employed
in new areas of sensing, actuation, and even user interaction.
\\ \ \\
There are two types of RFID sensing in use, Analog RFID sensing and Digital RFID sensing.
\textbf{Analog} RFID sensing chips perform an analog processing of the physical signals related
to the communication between the reader and the tag, with no dedicated sensing electronics.
The reader is able to obtain much more information about the target, more than just identification,
without the need for additional electronics.  
\textbf{Digital} RFID sensing is a method which has its tags are integrated with electronic components, such as sensory material,
analog-to-digital converters, and a microcrontroller, to make an integrated sensor module.
These systems are referred to as Computational RFID (CRFID). CRFID systems permit running
programs on embedded computers using only scavenged Radio Frequency (RF) energy. Battery
free, invisible sensing and computation is key to truly ubiquitous computing applications for
the IoT. The CRFID tag is used as a communication interface for transmitting data.
\\ \ \\
Both analog and digital RFID sensing can provide a variety of sensing capabilities to IoT systems.
The decision to use one or the other will depend on the particular application and its particular use-case.
\textbf{Analog} RFID sensing chips are in general, much cheaper than their \textbf{Digital} equivalents, but are 
much less accurate in their workings and have a critically lower dynamic range in sensing.
\textbf{Digital} RFID sensing or CRFID systems can produce
more accurate and very selective outcomes at the expense of a higher level of harvested energy.
\\ \ \\
But RFID chips are not as prevalent in domestic environments as RFIP scanners and readers are of a much higher cost than
sensors and tags. They are also made of elements that are relatively rare to obtain.
\\ \ \\
RFID sensing chips are of great use in the medical sector as they have many utilities.
Wearable and wireless devices allow efficient and continuous medical
monitoring. Wireless sensors do not limit the patient’s movements, thereby improving their quality
of life. Similarly, medical providers can gather data, facilitate inter-departmental communication,
and dispatch emergency care more efficiently.
RFID technology is already being widely adopted across the retail sector. RFID tags can
identify every product in a store with a unique identifying number; they reduce the need for human
resources, and eradicate human error by automating processes; they enable simultaneous product
scanning; offer real-time stock information; provide new ways of advertising; and increase the security
of the staff, equipment, and stock.   
RFID sensing is emerging in the field of neuro-engineering. Implantable
biomedical devices are one of the application areas of RFID because of the great
potential that wireless power and data communication capabilities, inherent in RFID, bring to this field.
Hence, the RFID physical layer is ideal for applications such as neural recording, where implanted
sensors do not require any source of energy of their own except for an external radio frequency field.
\\ \ \\
\textbf{WSN}
\\ \ \\
WSNs are collections of nodes implementing sensors that collect data in a distributed manner
and wirelessly transmit it to a main node. A WSN is mainly composed of sensing nodes, gateways
(base station or router), a coordinator, and a PC server. The sensing nodes collect the information from
their respective sensors and communicate with a PC server through the gateways. WSNs are widely
used in medical, environmental, military, and security applications.
Whilst RFID is used to identify and track devices, WSNs collaboratively gather and provide
information from their sensors. These two technologies can work together to exploit their advantages
and complement their limitations.
\\ \ \\
A WSN can use hundreds of sensors,
accompanied by gateways and a coordinating device, to sense the environmental or physical conditions
of a system, and to monitor or control it. Each node contains one or more sensors, which can be passive
or active. These sensors communicate with each other to transmit the information to a server PC
that manages the information of the entire network. Typical technologies and communication
standards used in WSN are WiFi and Bluetooth on the physical and Media Access Control (MAC) layers,
and ZigBee and 6LowPan protocols in the network, security, and application layers. The data rate balances the
energy consumed by the system and the scalability to interrogate a
higher number of nodes in the same amount of time. The data rate is directly related to the bandwidth:
working in GHz channels will ensure higher data rates than those of the licensed wireless technologies
(400 MHz, 800-900 MHz), because the latter use narrower channels than the ones in the GHz band.
Additionally, coverage can affect the scalability of the system, since more nodes would be needed.
But not only coverage affects it but also, the power of the device, the radio regulations, the coding,
the modulation, the properties of the radio propagation, and the topology of the network. A network
with a lower data rate covers a wider area at the expense of an increase in latency.
\\ \ \\
The following are some applications of WSNs used currently - 
Terrestrial sensors, which are deployed or
pre-planned in terrestrial environment; Underground sensors that can be deployed in caves, mines or
under the soil, something which limits the replacement of their batteries; Mobile sensors, which can
monitor the physical environment, the habitat or track a target; and multimedia sensors that can store,
save or process multimedia data such as audio or video, requiring high bandwidth, QoS and power.
Considering all these possibilities, WSNs have revolutionized and improved fields such as ecology,
transport, entertainment, health, and security among others. In particular, in road transport, WSNs are
used to gather data to gain an insight and provide services to manage traffic.
\\ \ \\
\textbf{Conclusion}
\\ \ \\
The paper does an exemplary job at conveying the utilities and workings of both RFID technology and WSNs,
and illustrates the importance of further research on the topic to advance the capabilities of said technologies.
Research and advancement in regards to the integration of both RFID and WSNs into devices and systems, including passive
and computational sensors and more robust and less power demanding networks would revolutionize the IoT industry.
\end{document}